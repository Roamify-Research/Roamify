%%
%% This is file `sample-sigconf-authordraft.tex',
%% generated with the docstrip utility.
%%
%% The original source files were:
%%
%% samples.dtx  (with options: `all,proceedings,bibtex,authordraft')
%% 
%% IMPORTANT NOTICE:
%% 
%% For the copyright see the source file.
%% 
%% Any modified versions of this file must be renamed
%% with new filenames distinct from sample-sigconf-authordraft.tex.
%% 
%% For distribution of the original source see the terms
%% for copying and modification in the file samples.dtx.
%% 
%% This generated file may be distributed as long as the
%% original source files, as listed above, are part of the
%% same distribution. (The sources need not necessarily be
%% in the same archive or directory.)
%%
%%
%% Commands for TeXCount
%TC:macro \cite [option:text,text]
%TC:macro \citep [option:text,text]
%TC:macro \citet [option:text,text]
%TC:envir table 0 1
%TC:envir table* 0 1
%TC:envir tabular [ignore] word
%TC:envir displaymath 0 word
%TC:envir math 0 word
%TC:envir comment 0 0
%%
%%
%% The first command in your LaTeX source must be the \documentclass
%% command.
%%
%% For submission and review of your manuscript please change the
%% command to \documentclass[manuscript, screen, review]{acmart}.
%%
%% When submitting camera ready or to TAPS, please change the command
%% to \documentclass[sigconf]{acmart} or whichever template is required
%% for your publication.
%%
%%
\documentclass[sigconf,authordraft]{acmart}

%%
%% \BibTeX command to typeset BibTeX logo in the docs
\AtBeginDocument{%
  \providecommand\BibTeX{{%
    Bib\TeX}}}

%% Rights management information.  This information is sent to you
%% when you complete the rights form.  These commands have SAMPLE
%% values in them; it is your responsibility as an author to replace
%% the commands and values with those provided to you when you
%% complete the rights form.
\setcopyright{acmlicensed}
\copyrightyear{2024}
\acmYear{2024}
\acmDOI{XXXXXXX.XXXXXXX}

%% These commands are for a PROCEEDINGS abstract or paper.
\acmConference[Conference acronym 'XX]{Make sure to enter the correct
  conference title from your rights confirmation emai}{June 03--05,
  2018}{Woodstock, NY}
%%
%%  Uncomment \acmBooktitle if the title of the proceedings is different
%%  from ``Proceedings of ...''!
%%
%%\acmBooktitle{Woodstock '18: ACM Symposium on Neural Gaze Detection,
%%  June 03--05, 2018, Woodstock, NY}
\acmISBN{978-1-4503-XXXX-X/18/06}


%%
%% Submission ID.
%% Use this when submitting an article to a sponsored event. You'll
%% receive a unique submission ID from the organizers
%% of the event, and this ID should be used as the parameter to this command.
%%\acmSubmissionID{123-A56-BU3}

%%
%% For managing citations, it is recommended to use bibliography
%% files in BibTeX format.
%%
%% You can then either use BibTeX with the ACM-Reference-Format style,
%% or BibLaTeX with the acmnumeric or acmauthoryear sytles, that include
%% support for advanced citation of software artefact from the
%% biblatex-software package, also separately available on CTAN.
%%
%% Look at the sample-*-biblatex.tex files for templates showcasing
%% the biblatex styles.
%%

%%
%% The majority of ACM publications use numbered citations and
%% references.  The command \citestyle{authoryear} switches to the
%% "author year" style.
%%
%% If you are preparing content for an event
%% sponsored by ACM SIGGRAPH, you must use the "author year" style of
%% citations and references.
%% Uncommenting
%% the next command will enable that style.
%%\citestyle{acmauthoryear}


%%
%% end of the preamble, start of the body of the document source.
\begin{document}

%%
%% The "title" command has an optional parameter,
%% allowing the author to define a "short title" to be used in page headers.
\title{Roamify: Evaluating a Google Chrome Extension for Enhancing User Experience in Itinerary Planning}

%%
%% The "author" command and its associated commands are used to define
%% the authors and their affiliations.
%% Of note is the shared affiliation of the first two authors, and the
%% "authornote" and "authornotemark" commands
%% used to denote shared contribution to the research.
\author{Vikranth Udandarao}
\affiliation{%
  \institution{IIIT Delhi}
  \department{Computer Science Engineering Dept}
  \city{New Delhi}
  \country{India}}
\email{vikranth22570@iiitd.ac.in}

\author{Noel Abraham Tiju}
\affiliation{%
  \institution{IIIT Delhi}
  \department{Computer Science Engineering Dept}
  \city{New Delhi}
  \country{India}}
\email{noel22338@iiitd.ac.in}

\author{Muthuraj Vairamuthu}
\affiliation{%
  \institution{IIIT Delhi}
  \department{Computer Science Engineering Dept}
  \city{New Delhi}
  \country{India}}
\email{muthuraj22307@iiitd.ac.in}

\author{Harsh Mistry}
\affiliation{%
  \institution{IIIT Delhi}
  \department{Computer Science Engineering Dept}
  \city{New Delhi}
  \country{India}}
\email{harsh22200@iiitd.ac.in}

\author{Armaan Singh}
\affiliation{%
  \institution{IIIT Delhi}
  \department{Computer Science Engineering Dept}
  \city{New Delhi}
  \country{India}}
\email{armaan22096@iiitd.ac.in}

\author{Dhruv Kumar}
\affiliation{%
  \institution{IIIT Delhi}
  \department{Computer Science Engineering Dept}
  \city{New Delhi}
  \country{India}}
\email{dhruv.kumar@iiitd.ac.in}


%%
%% By default, the full list of authors will be used in the page
%% headers. Often, this list is too long, and will overlap
%% other information printed in the page headers. This command allows
%% the author to define a more concise list
%% of authors' names for this purpose.
\renewcommand{\shortauthors}{Trovato et al.}

%%
%% The abstract is a short summary of the work to be presented in the
%% article.
\begin{abstract}
  Travel planning often presents significant challenges and consumes considerable time. Existing LLM-powered tools, such as ChatGPT-4, suffer from limitations due to outdated training data. To address these shortcomings, we developed Roamify, an AI-driven itinerary planning plugin that streamlines the travel planning process by leveraging blog data and tailoring itineraries to individual user preferences. This paper presents Roamify's innovative methods, which transform vacation planning from a logistical burden into an exciting and personalized adventure. Through our user study, we identified three critical design considerations for future travel assistants: \textbf{D1)} integrating web scraping techniques to gather the latest news articles about a destination and incorporating them into itineraries, \textbf{D2)} aggregating attractions and ratings from diverse blog sources to enhance itinerary recommendations, and \textbf{D3)} leveraging user preferences to generate customized travel experiences. Our findings suggest that Roamify has the potential to revolutionize how users plan their trips, offering a more efficient and enjoyable process.

\end{abstract}

%%
%% The code below is generated by the tool at http://dl.acm.org/ccs.cfm.
%% Please copy and paste the code instead of the example below.
%%
\begin{CCSXML}
<ccs2012>
   <concept>
       <concept_id>10003120.10003138.10003140</concept_id>
       <concept_desc>Human-centered computing~Ubiquitous and mobile computing systems and tools</concept_desc>
       <concept_significance>500</concept_significance>
       </concept>
   <concept>
       <concept_id>10010405.10010476.10010479</concept_id>
       <concept_desc>Applied computing~Cartography</concept_desc>
       <concept_significance>500</concept_significance>
       </concept>
 </ccs2012>
\end{CCSXML}

\ccsdesc[500]{Human-centered computing~Ubiquitous and mobile computing systems and tools}
\ccsdesc[500]{Applied computing~Cartography}

%%
%% Keywords. The author(s) should pick words that accurately describe
%% the work being presented. Separate the keywords with commas.
\keywords{large language models, AI assistants, travel planning assistants, generative AI}
%%
%% This command processes the author and affiliation and title
%% information and builds the first part of the formatted document.
\maketitle

\section{Introduction}
In the past, travel was a luxury for the wealthy elite, but advancements in transportation have made it accessible to all. People now travel for various reasons—work, leisure, reunions, or exploration—creating a new challenge: deciding what to explore upon arrival. Traditionally, a family planning a vacation would rely on pre-arranged packages from travel agencies. However, the rise of online content, such as blogs and videos, has empowered travelers to plan their trips with greater flexibility and independence, moving away from conventional packages.

Large Language Models (LLMs) have revolutionized AI research by paving the way for complex applications and software across diverse sectors. In tourism, LLMs hold immense potential to enhance the travel experience by incorporating personalized travel recommendations, dynamic itinerary planning, and customized travel content[1]. While LLMs are extensively used in the modern era, concerns have been raised regarding privacy issues, particularly the potential for data leaks and the recovery of sensitive user information from inputs[2]. This has created a need for solutions prioritizing generalization, ensuring that personal details such as names and other identifying information are not retained or exposed.

Building on the need for generalized, privacy-conscious solutions, Roamify, through its 'Roaming Redefined' initiative, harnesses the capabilities of LLMs to deliver a comprehensive platform for modern travel planning. Catering to a diverse audience—including youngsters, teenagers, college students, and families—Roamify tackles the common challenge of exploring new destinations without the burden of extensive planning. By employing LLMs, the application generates AI-driven itineraries with carefully curated personalized recommendations designed to protect user privacy.

Through comprehensive surveys and interviews, we gained valuable insights into potential users' travel planning habits and preferences. Our findings showed a clear preference for personalized and flexible travel options, particularly among young adults and professionals. Millennials and Generation Z favor AI-driven tools, while Generation X gravitates toward traditional travel agencies. Hence, it underscores the demand for a user-friendly, AI-powered travel solution such as Roamify, designed to cater to a wide range of age groups with a strong emphasis on personalization and efficiency.

To effectively understand the implications of AI-powered tools in the tourism industry, we guide this paper with the following research questions:

\begin{itemize}
    \item \textbf{RQ1 - Involvement of user preferences:} What role do user preferences of genres like Historical, Amusement, Natural, and others play in planning travel itineraries?
    \item \textbf{RQ2 - Itinerary responses:} How effective and practical are the results obtained from the application?
    \item \textbf{RQ3 - Itinerary planning practices:} How do users currently approach itinerary planning, and in what ways does Roamify simplify and enhance this process by providing personalized, AI-driven travel recommendations?
\end{itemize}

By addressing these research questions, this paper aims to elucidate the methodology behind developing the Roamify application, explore its implications within the tourism industry, and consider future design considerations and potential impacts. Through this exploration, we intend to provide a comprehensive understanding of how AI-driven tools like Roamify can transform travel planning, highlighting such technology's current and future roles in enhancing user experiences.

\section{Related Work}
As noted in the introduction, the ``\verb|acmart|'' document class can
be used to prepare many different kinds of documentation --- a
double-anonymous initial submission of a full-length technical paper, a
two-page SIGGRAPH Emerging Technologies abstract, a ``camera-ready''
journal article, a SIGCHI Extended Abstract, and more --- all by
selecting the appropriate {\itshape template style} and {\itshape
  template parameters}.

This document will explain the major features of the document
class. For further information, the {\itshape \LaTeX\ User's Guide} is
available from
\url{https://www.acm.org/publications/proceedings-template}.

\subsection{LLMs in Tourism Industry}


%%
%% The acknowledgments section is defined using the "acks" environment
%% (and NOT an unnumbered section). This ensures the proper
%% identification of the section in the article metadata, and the
%% consistent spelling of the heading.
\begin{acks}
To Robert, for the bagels and explaining CMYK and color spaces.
\end{acks}

%%
%% The next two lines define the bibliography style to be used, and
%% the bibliography file.
\bibliographystyle{ACM-Reference-Format}
\bibliography{sample-base}


%%
%% If your work has an appendix, this is the place to put it.
\appendix

\section{Research Methods}

\subsection{Part One}

Lorem ipsum dolor sit amet, consectetur adipiscing elit. Morbi
malesuada, quam in pulvinar varius, metus nunc fermentum urna, id
sollicitudin purus odio sit amet enim. Aliquam ullamcorper eu ipsum
vel mollis. Curabitur quis dictum nisl. Phasellus vel semper risus, et
lacinia dolor. Integer ultricies commodo sem nec semper.

\subsection{Part Two}

Etiam commodo feugiat nisl pulvinar pellentesque. Etiam auctor sodales
ligula, non varius nibh pulvinar semper. Suspendisse nec lectus non
ipsum convallis congue hendrerit vitae sapien. Donec at laoreet
eros. Vivamus non purus placerat, scelerisque diam eu, cursus
ante. Etiam aliquam tortor auctor efficitur mattis.

\end{document}
\endinput
%%
%% End of file `sample-sigconf-authordraft.tex'.
